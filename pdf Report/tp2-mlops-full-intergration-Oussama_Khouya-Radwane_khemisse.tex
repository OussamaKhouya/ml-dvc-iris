\documentclass[11pt,a4paper]{article}
\usepackage[utf8]{inputenc}
\usepackage[T1]{fontenc}
\usepackage{lmodern}
\usepackage[french]{babel}
\usepackage{geometry}
\usepackage{graphicx}
\usepackage{grffile}
\usepackage{listings}
\usepackage{xcolor}
\usepackage{hyperref}

\geometry{margin=2.5cm}
\hypersetup{colorlinks=true, linkcolor=blue, urlcolor=blue}

\lstset{
  basicstyle=\ttfamily\small,
  keywordstyle=\color{blue},
  commentstyle=\color{gray},
  stringstyle=\color{teal},
  frame=single,
  breaklines=true
}

\title{TP2 -- Intégration MLOps complète\\
\small (GitHub Actions, DVC, CML, Docker, Google Drive)}
\author{Oussama Khouya \and Radwane Khemisse \and Achrafe Elalaoui}
\date{\today}

\begin{document}
\maketitle

\section{Introduction}
Projet \texttt{ml-dvc-iris} dérivé du TP2 (pipeline DVC : prepare $\rightarrow$ train $\rightarrow$ evaluate). Objectifs : automatiser la chaîne MLOps (DVC + GitHub Actions + CML), stocker les artefacts sur Google Drive, assurer la reproductibilité (\texttt{dvc repro}), préparer la containerisation et la promotion du meilleur modèle.

\section{Architecture MLOps mise en place}
\begin{itemize}
  \item Pipeline DVC (\texttt{dvc.yaml}) avec trois stages (\autoref{lst:dvc}).
  \item Données/artefacts suivis : \texttt{data/iris.csv}, \texttt{data/iris\_preprocessed.csv}, \texttt{models/random\_forest.pkl}, \texttt{metrics/*.json}.
  \item Remote DVC Google Drive : \texttt{new\_gdrive} configuré via secrets GitHub.
  \item CI/CD : workflow GitHub Actions \texttt{mlops-pipeline.yaml} (checkout, install deps, config remote, \texttt{dvc pull}, fallback dataset, \texttt{dvc repro}, rapport CML, commentaire PR).
  \item Reporting : \texttt{scripts/generate\_cml\_report.py} et commentaire CML automatique.
  \item Containerisation : Dockerfile à ajouter (build/run non réalisés).
  \item Promotion modèle : stage \texttt{deploy} et \texttt{models/production\_model.pkl} à compléter.
\end{itemize}

\begin{lstlisting}[language=YAML,caption={Pipeline DVC},label={lst:dvc}]
stages:
  prepare:
    cmd: python scripts/preprocess.py
    deps:
      - scripts/preprocess.py
      - data/iris.csv
    outs:
      - data/iris_preprocessed.csv

  train:
    cmd: python src/train.py
    deps:
      - src/train.py
      - data/iris_preprocessed.csv
      - params.yaml
    outs:
      - models/random_forest.pkl
      - metrics/train_metrics.json

  evaluate:
    cmd: python src/evaluate.py
    deps:
      - src/evaluate.py
      - models/random_forest.pkl
      - data/iris_preprocessed.csv
    outs:
      - metrics/eval_metrics.json
\end{lstlisting}

\begin{figure}[h]
  \centering
  \includegraphics[width=0.82\linewidth]{../snapshots/dvc repro.png}
  \caption{Exécution locale du pipeline \texttt{dvc repro}}
  \label{fig:dvc-repro}
\end{figure}

\section{CI GitHub Actions et CML}
\subsection{Dépendances et rapport}
\begin{itemize}
  \item Dépendances ML/DVC/CML dans \texttt{requirements.txt} (\autoref{fig:req}).
  \item Rapport Markdown généré par \texttt{scripts/generate\_cml\_report.py} (\autoref{fig:gen-cmd}, \autoref{fig:gen-report}).
\end{itemize}

\begin{figure}[h]
  \centering
  \includegraphics[width=0.7\linewidth]{../snapshots/requirements.png}
  \caption{Dépendances Python}
  \label{fig:req}
\end{figure}

\begin{figure}[h]
  \centering
  \includegraphics[width=0.6\linewidth]{../snapshots/generate-report-command.png}
  \caption{Génération du rapport CML}
  \label{fig:gen-cmd}
\end{figure}

\begin{figure}[h]
  \centering
  \includegraphics[width=0.78\linewidth]{../snapshots/generated-report.png}
  \caption{Aperçu de \texttt{reports/cml\_report.md}}
  \label{fig:gen-report}
\end{figure}

\subsection{Workflow GitHub Actions}
\begin{itemize}
  \item Triggers : \texttt{push}, \texttt{pull\_request}.
  \item Étapes : checkout, Python 3.11, install deps (\texttt{dvc[gdrive]}, \texttt{cml}), config remote GDrive, \texttt{dvc pull}, fallback \texttt{scripts/download\_iris.py} si besoin, \texttt{dvc repro}, génération du rapport, \texttt{cml comment create}.
  \item Succès du workflow (\autoref{fig:ci-ok}) et commentaire CML en PR (\autoref{fig:cml-comment}).
\end{itemize}

\begin{lstlisting}[language=YAML,caption={Extrait du workflow mlops-pipeline.yaml},label={lst:workflow}]
on:
  push:
  pull_request:
jobs:
  run:
    runs-on: ubuntu-latest
    steps:
      - uses: actions/checkout@v4
      - uses: actions/setup-python@v5
        with: { python-version: "3.11" }
      - uses: iterative/setup-cml@v2
      - run: |
          python -m pip install --upgrade pip setuptools wheel
          pip install -r requirements.txt
          pip install "dvc[gdrive]==3.63.0"
      - name: Configure DVC remote (OAuth)
        run: |
          echo '${{ secrets.GDRIVE_CREDENTIALS_JSON }}' > gdrive_user_credentials.json
          dvc remote modify --local new_gdrive gdrive_user_credentials_file gdrive_user_credentials.json
          dvc remote modify --local new_gdrive gdrive_client_id "${{ secrets.GDRIVE_CLIENT_ID }}"
          dvc remote modify --local new_gdrive gdrive_client_secret "${{ secrets.GDRIVE_CLIENT_SECRET }}"
      - run: dvc pull -v
      - run: if [ ! -f data/iris.csv ]; then python scripts/download_iris.py; fi
      - run: dvc repro
      - run: |
          python3 scripts/generate_cml_report.py
          cml comment create reports/cml_report.md
\end{lstlisting}

\begin{figure}[h]
  \centering
  \includegraphics[width=0.8\linewidth]{../snapshots/successful-workflow.png}
  \caption{Exécution réussie du workflow GitHub Actions}
  \label{fig:ci-ok}
\end{figure}

\begin{figure}[h]
  \centering
  \includegraphics[width=0.75\linewidth]{../snapshots/cml-comment.png}
  \caption{Commentaire CML généré dans la Pull Request}
  \label{fig:cml-comment}
\end{figure}

\section{Remote DVC Google Drive}
\begin{itemize}
  \item Remote \texttt{new\_gdrive} pointant vers l'ID Drive, authentifié via secrets OAuth.
  \item Synchronisation vérifiée : \texttt{dvc pull} (\autoref{fig:dvc-pull}) et \texttt{dvc push} (\autoref{fig:dvc-push}).
\end{itemize}

\begin{lstlisting}[language=ini,caption={Extrait .dvc/config},label={lst:dvc-config}]
[core]
    remote = new_gdrive

[remote "new_gdrive"]
    url = gdrive://1qnTG-xYstcnUbljTv94pp2izEqLpBP-o
    gdrive_use_service_account = true
    gdrive_service_account_json_file_path = /home/oldhome/pc/enset/_S3/DevOps/dvc/gdrive_user_credentials.json
\end{lstlisting}

\begin{figure}[h]
  \centering
  \includegraphics[width=0.78\linewidth]{../snapshots/dvc pull.png}
  \caption{Récupération des artefacts par \texttt{dvc pull}}
  \label{fig:dvc-pull}
\end{figure}

\begin{figure}[h]
  \centering
  \includegraphics[width=0.78\linewidth]{../snapshots/dvc push up to date.png}
  \caption{\texttt{dvc push} indiquant un cache distant à jour}
  \label{fig:dvc-push}
\end{figure}

\section{Containerisation Docker}
\begin{itemize}
  \item Attendu : Dockerfile (base \texttt{python:3.11-slim}, install deps, copie projet, \texttt{CMD ["dvc","repro"]}).
  \item Statut : non réalisé, aucune capture de build/run dans \texttt{./snapshots}.
  \item À faire : ajouter Dockerfile, exécuter \texttt{docker build/run}, produire la capture.
\end{itemize}

\section{Promotion du meilleur modèle}
\begin{itemize}
  \item Attendu : script \texttt{scripts/deploy.py} et stage \texttt{deploy} produisant \texttt{models/production\_model.pkl}.
  \item Statut : non implémenté, fichier \texttt{models/production\_model.pkl} non généré.
  \item À faire : comparer \texttt{metrics/eval\_metrics.json} à \texttt{metrics/best\_metrics.json}, copier \texttt{models/random\_forest.pkl} en cas de meilleure accuracy.
\end{itemize}

\section{Difficultés rencontrées et solutions}
\begin{itemize}
  \item Authentification GDrive : tokens corrompus/type invalide. Solution : reconfigurer \texttt{new\_gdrive} avec OAuth via secrets et valider \texttt{dvc pull/push}.
  \item Build PyYAML en CI : échec de wheel. Solution : pinner \texttt{pyyaml==6.0.2} et installer \texttt{libyaml-dev}+\texttt{build-essential}.
  \item Données brutes absentes : \texttt{iris.csv} manquante au début. Solution : \texttt{dvc add data/iris.csv} puis \texttt{dvc push}; fallback \texttt{scripts/download\_iris.py} dans la CI.
  \item Plugin GDrive manquant en CI : erreur “dvc-gdrive”. Solution : installer explicitement \texttt{dvc[gdrive]==3.63.0} dans le workflow.
  \item Reste à faire : Dockerfile/build-run, stage \texttt{deploy} et génération de \texttt{models/production\_model.pkl}.
\end{itemize}

\section{Conclusion}
Chaîne MLOps quasi complète : pipeline DVC reproductible, remote Google Drive fonctionnel, workflow GitHub Actions + CML opérationnel avec rapport et commentaire PR. Travaux restants : containerisation Docker (build/run), ajout du stage \texttt{deploy} pour produire \texttt{models/production\_model.pkl} et finaliser la promotion du meilleur modèle.

\end{document}
